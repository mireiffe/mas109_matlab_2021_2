\documentclass[11pt,a4paper]{article}
%\usepackage{hangul}
\usepackage{amsfonts,dsfont}
\usepackage{color,soul}
\usepackage{tcolorbox}
\usepackage{amsmath,amssymb,tabularx,graphicx}
\usepackage{systeme}
\pagestyle{empty}
\textheight 217mm
\textwidth 159mm
\hoffset = -15mm
\voffset = -7mm

\begin{document}
\centerline{\Large\bf MATLAB Midterm}
\vskip .05 in
\hrule
\vskip .1 in
{\noindent \bf Introduction to Linear Algebra \hfill Spring, 2021}
\vskip 0.2 in

\begin{enumerate}
\item[10.] (MATLAB Programming Problem)

\begin{enumerate}
    \item[(1)] (10 points)
    
    \vspace{1mm}
    
    The following is contents of a MATLAB script file \verb"operations.m":

    \vspace{3mm}

    \begin{verbatim}
% Declare two matrices, A and B.
A = [5, -2, -3; -5, 1, 1; 0, 1, 7];
B = [3, 1, 2; -1, 1, -2; 1, 1, 4];

% ----------- operations ------------
M1 = A * B;     M2 = A .* B;
P1 = A^2;       P2 = A.^2;
D1 = A / B;     D2 = A ./ B;
% -----------------------------------

% some calculations.
a = sum(M1(:)) * sum(M2(2, :));
b = min(P1(:)) - max(P2, 2);
c = sum(D1 > D2, `a');

answer = a * c + b;     % compute `answer'.
disp(answer)            % print the `answer' on `Command Window'.
    \end{verbatim}

    What is the number printed when running this script file, \verb"operations.m" ?
    % \begin{align*}
        % 1.~-478 && 2.~-103 && 3.~322 && 4.~513 && 5.~558
    % \end{align*}

    \vspace{3mm}

    \item[(2)] (10 potins in totall, 2 points for each of blank.)
    
    \vspace{1mm}
    
    Let $f$ be a function obtained by rotating the function $z = e^{-x^2 -y}\sin{2x^4} + 3\cos{xy}$ 
    counterclockwise 17 degrees about the z axis. In this problem, we write a script file {\verb"rotFunc.m"} 
    to find and draw the surface represented by the function $f$ on a domain
    $D = \left\{(x,\,y)|-1\le x \le 1,\, -1\le y \le 1\right\}$ with increments of $0.01$ and $0.05$, respectively. 
    Fill in the blanks (1) - (5).

    \vspace{2mm}

    \begin{verbatim}
% ---------- The following is the script file `rotFunc.m'.----------
hx = 0.01;   hy = 0.05; % Set the increment as described in the problem.

x = -1:___(1)___:1;     % A vector x of specified interval and increment.
y = -1:hy:1;            % A vector y of specified interval and increment.

% Construct a grid over the domain D in the problem.
[X, Y] = ___(2)___(x, y);

theta = ___(3)___;      % Set the theta as an appropriate radian value.
C = cos(theta);         % Store the value of cos(theta)
S = sin(theta);         % Store the value of sin(theta)

% Construct clockwise rotated grids, rotX and rotY.
rotX = C.*X + S.*Y;
rotY = -S.*X + C.*Y;

% Evaluate the function f(x,y) discribed in the problem.
fxy = ___(4)___;

figure(1);              % Open the figure 1.
___(5)___(X, Y, fxy);   % Draw the surface.
grid on;                % Turn the grid on.
    \end{verbatim}
\end{enumerate}

\end{enumerate}

\newpage

\textit{Solution.}
\begin{enumerate}
    \item[(1)] answer : $558$
    
    \vspace{2mm}
    
    In the first two lines, matrices $A$ and $B$ are declared:
    \begin{align*}
        A =
        \begin{bmatrix}
            5 & -2 & -3\\ -5 & 1 & 1\\ 0 & 1 & 7
        \end{bmatrix}
        ,\quad
        B = 
        \begin{bmatrix}
            3 & 1 & 2\\ -1 & 1 & -2\\ 1 & 1 & 4
        \end{bmatrix}.
    \end{align*}
    In the MATLAB, dot(\verb".") operators do element-wise operations. Furthermore, \verb"A/B" is the same as \verb"A * inv(B)". Thus in the above operations,
    \begin{align*}
        M1 =
        \begin{bmatrix}
            14 & 0 & 2\\ -15 & -3 & -8\\ 6 & 8 & 26
        \end{bmatrix}
        \quad &\mbox{and}\quad
        M2 = 
        \begin{bmatrix}
            15 & -2 & -6\\ 5 & 1 & -2\\ 0 & 1 & 28
        \end{bmatrix},\\
        P1 =
        \begin{bmatrix}
            35 & -15 & -38\\ -30 & 12 & 23\\ -5 & 8 & 50
        \end{bmatrix}
        \quad &\mbox{and}\quad
        P2 = 
        \begin{bmatrix}
            25 & 4 & 9\\ 25 & 1 & 1\\ 0 & 1 & 49
        \end{bmatrix},\\
        D1 =
        \begin{bmatrix}
            2 & -3/2 & -5/2\\
            -15/8 & 9/8 & 7/4\\
            -3/4 & -1/4 & 2
        \end{bmatrix}
        \quad &\mbox{and}\quad
        D2 = 
        \begin{bmatrix}
            5/3 & -2 & -3/2\\
            5 & 1 & -1/2\\
            0 & 1 & 7/4
        \end{bmatrix}.
    \end{align*}

    Then, variables $a=120$, $b=-42$ and $c=5$ are easily calculated. Therefore, $a * c + b = 558$ will be printed.

    \item[(2)] ~
    \begin{enumerate}
        \item[(1)] \verb"hx"
        \item[(2)] \verb"meshgrid"
        \item[(3)] \verb"17 * pi / 180"
        \item[(4)] \verb"exp(-rotX.^2 -rotY) .* sin(2 * rotX.^4) + 3 * cos(rotX .* rotY)"
        \item[(5)] \verb"mesh" \textit{or} \verb"surf"
    \end{enumerate}
\end{enumerate}

\end{document}

