\documentclass[11pt,a4paper]{article}
%\usepackage{hangul}
\usepackage{amsfonts,dsfont}
\usepackage{color,soul}
\usepackage{tcolorbox}
\usepackage{amsmath,amssymb,tabularx,graphicx}
\usepackage{systeme}
\pagestyle{empty}
\textheight 217mm
\textwidth 159mm
\hoffset = -15mm
\voffset = -7mm

\begin{document}
\centerline{\Large\bf MATLAB Midterm}
\vskip .05 in
\hrule
\vskip .1 in
{\noindent \bf Introduction to Linear Algebra \hfill Spring, 2021}
\vskip 0.2 in

\begin{enumerate}
\item[10.] (MATLAB Programming Problem)

\begin{enumerate}
    \item[(1)] (10 points)
    
    \vspace{1mm}
    
    The following is contents of a MATLAB script file \verb"operations.m":

    \vspace{3mm}

    \begin{verbatim}
% Declare two matrices, A and B.
A = [5, -2, -3; -5, 1, 1; 0, 1, 7];
B = [3, 1, 2; -1, 1, -2; 1, 1, 4];

%--------- operations ----------
M1 = A * B;     M2 = A .* B;
D1 = A / B;     D2 = A ./ B;
% -------------------------------

% some calculations.
a = min(M1(:)) * sum(M2(2, :));
b = sum(D1 > D2, `a');

answer = a + b;         % compute `answer'.
disp(answer)            % print the `answer' on `Command Window'.
    \end{verbatim}

    Choose the number printed when running this script file, \verb"operations.m".
    \begin{align*}
        \mbox{(a)}~-120 && \mbox{(b)}~-55 && \mbox{(c)}~5 && \mbox{(d)}~125 && \mbox{(e)}~536
    \end{align*}

    \vspace{3mm}

    \item[(2)] (10 potins in total, 2 points for each of blank.)
    
    \vspace{1mm}
    
    Let $g(x,\, y)$ be a function obtained by rotating the function $f(x,\, y) = e^{-x^2 -y}\sin{2x^4} + 3\cos{xy}$ 
    counterclockwise 17 degrees about the $z$-axis. In this problem, we write a script file {\verb"rotFunc.m"} 
    to find and draw the surface represented by the function $g(x,\, y)$ on a domain
    $D = \left\{(x,\,y)|-1\le x \le 1,\, -1\le y \le 1\right\}$ 
    with increments of $0.01$ for $x$ and $0.05$ for $y$. Fill in the blanks (1) - (5).

    \vspace{2mm}

    \begin{verbatim}
% ---------- The following is the script file `rotFunc.m'.----------
hx = 0.01;   hy = 0.05; % Set the increment as described in the problem.

x = -1:___(1)___:1;     % A vector x of specified interval and increment.
y = -1:hy:1;            % A vector y of specified interval and increment.

% Construct a grid over the domain D in the problem.
[X, Y] = ___(2)___(x, y);

theta = ___(3)___;      % Set the theta as an appropriate radian value.
C = cos(theta);         % Store the value of cos(theta)
S = sin(theta);         % Store the value of sin(theta)

% Construct clockwise rotated grids, rotX and rotY.
rotX = C.*X - ___(4)___.*Y;
rotY = ___(4)___.*X + C.*Y;

% Evaluate the function g(x,y) discribed in the problem.
gxy = exp(-rotX.^2-rotY).*sin(2*rotX.^4) + 3*cos(rotX.*rotY); 

figure(1);              % Open the figure 1.
___(5)___(X, Y, gxy);   % Draw the surface.
grid on;                % Turn the grid on.
    \end{verbatim}

    \begin{itemize}
        \item[(a)]
            (1) \verb"hx", (2) \verb"meshgrid", (3) \verb"-17 * pi / 180", (4) \verb"-S", (5) \verb"surf"
        \item[(b)]
            (1) \verb"hx", (2) \verb"meshgrid", (3) \verb"17 * pi / 180", (4) \verb"S", (5) \verb"mesh"
        \item[(c)]
            (1) \verb"hx", (2) \verb"meshgrid", (3) \verb"-17 * pi / 180", (4) \verb"S", (5) \verb"plot3"
        \item[(d)]
            (1) \verb"hy", (2) \verb"grid on", (3) \verb"-17 * pi / 180", (4) \verb"-S", (5) \verb"mesh"
        \item[(e)]
            (1) \verb"hx", (2) \verb"meshgrid", (3) \verb"17 * pi / 180", (4) \verb"-S", (5) \verb"mesh"
    \end{itemize}

\end{enumerate}

\end{enumerate}

\newpage

\textit{Solution.}
\begin{enumerate}
    \item[(1)] answer : (b) $-55$
    
    \vspace{2mm}
    
    In the first two lines, matrices $A$ and $B$ are declared:
    \begin{align*}
        A =
        \begin{bmatrix}
            5 & -2 & -3\\ -5 & 1 & 1\\ 0 & 1 & 7
        \end{bmatrix}
        ,\quad
        B = 
        \begin{bmatrix}
            3 & 1 & 2\\ -1 & 1 & -2\\ 1 & 1 & 4
        \end{bmatrix}.
    \end{align*}
    In the MATLAB, dot(\verb".") operators do element-wise operations. Furthermore, \verb"A/B" is the same as \verb"A * inv(B)". Thus in the above operations,
    \begin{align*}
        M1 =
        \begin{bmatrix}
            14 & 0 & 2\\ -15 & -3 & -8\\ 6 & 8 & 26
        \end{bmatrix}
        \quad &\mbox{and}\quad
        M2 = 
        \begin{bmatrix}
            15 & -2 & -6\\ 5 & 1 & -2\\ 0 & 1 & 28
        \end{bmatrix},\\
        D1 =
        \begin{bmatrix}
            2 & -3/2 & -5/2\\
            -15/8 & 9/8 & 7/4\\
            -3/4 & -1/4 & 2
        \end{bmatrix}
        \quad &\mbox{and}\quad
        D2 = 
        \begin{bmatrix}
            5/3 & -2 & -3/2\\
            5 & 1 & -1/2\\
            0 & 1 & 7/4
        \end{bmatrix}.
    \end{align*}

    Then, variables $a=-60$ and $b=5$ are easily calculated. Therefore, $a + b = -55$ will be printed.

    \item[(2)] answer : (e)
    \begin{enumerate}
        \item[(1)] \verb"hx"
        \item[(2)] \verb"meshgrid"
        \item[(3)] \verb"17 * pi / 180"
        \item[(4)] \verb"-S"
        \item[(5)] \verb"mesh" \textit{or} \verb"surf"
    \end{enumerate}
\end{enumerate}

\end{document}

