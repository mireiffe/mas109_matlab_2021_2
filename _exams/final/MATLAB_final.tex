\documentclass[11pt,a4paper]{article}
%\usepackage{hangul}
\usepackage{amsfonts,dsfont}
\usepackage{color,soul}
\usepackage{tcolorbox}
\usepackage{amsmath,amssymb,tabularx,graphicx}
\usepackage{systeme}
\pagestyle{empty}
\textheight 217mm
\textwidth 159mm
\hoffset = -15mm
\voffset = -7mm

\begin{document}
\centerline{\Large\bf MATLAB Final}
\vskip .05 in
\hrule
\vskip .1 in
{\noindent \bf Introduction to Linear Algebra \hfill Spring, 2021}
\vskip 0.2 in

\begin{enumerate}
\item[10.] (MATLAB Programming Problem)

\begin{enumerate}
    \item[(1)] (10 points, 2 points for each)
    
    \vspace{1mm}
    Let $A$ be a $7 \times 8$ matrix given by
    \begin{align*}
        A =
        \begin{bmatrix}
            1 &  8 & 0 & 0 & 0 & 0 & 0 & 50\\
            2 &  9 & 0 & 0 & 0 & 0 & 44 & 51\\
            3 & 10 & 0 & 0 & 0 & 0 & 45 & 52\\
            4 & 11 & 0 & 0 & 0 & 0 & 46 & 53\\
            5 & 12 & 0 & 0 & 0 & 0 & 47 & 54\\
            6 & 13 & 0 & 0 & 0 & 0 & 48 & 55\\
            7 & 0 & 0 & 0 & 0 & 0 & 49 & 56\\
        \end{bmatrix}.
    \end{align*}

    Choose \textbf{True} for codes that generate the same matrix $A$ as above, or \textbf{False} for codes that do not.

    \vspace{2mm}
    
    \begin{itemize}
        \item[(a)] 
        \verb"A = reshape(1:56, 7, 8);"\\
        \verb"A(14 <= A & 43 >= A) = 0;"

        \item[(b)] 
        \verb"A = reshape(1:56, 7, 8);"\\
        \verb"A = A .* ((14 > A) + (43 < A));"
        
        \item[(c)]
        \verb"A(1:13) = 1:13;   A(44:56) = 44:56;"\\
        \verb"A = reshape(A, 7, 8);"
        
        \item[(d)] 
        \verb"A = reshape(1:56, 7, 8);"\\
        \verb"A = A(14 <= A) .* A(43 >= A);"
        
        \item[(e)] 
        \verb"A = zeros(7, 8);"\\
        \verb"A(:) = [1:13, zeros(1, 30), 44:56];"
    \end{itemize}

    \vspace{3mm}
    \newpage

    \item[(2)] (10 potins)
    
    \vspace{1mm}
    
    Let $A$ be an $m \times n$ matrix with full column rank. 
    The following MATLAB function performs a Gram-Schmidt process and produces 
    an $m \times n$ orthogonal matrix $Q$ whose columns form an orthonormal basis for col$(A)$. 
    Fill in the blanks (1) - (5).\\
    (In this problem, assume that the input $A$ is a matrix 
    that gives the same result for the both of classical Gram-Schmidt
     and the modified Gram-Schmidt.)

    \vspace{2mm}

    \begin{verbatim}
% ------- The following is the script file `GramSchmidt.m'. -------
___(1)___ Q = GramSchmidt(A)
[m, n] = size(A);

% Initialize the matrix Q as an m*n zero matrix.
Q = zeros(m, n);
for i = 1 : n
    % v starts with a column of A.
    v = ___(2)___;
    for j = 1 : i-1
        % Subtract orthogonal projections of v onto the subspaces
        % spanned by the previously generated orthonormal vectors.
        q = ___(3)___;
        v = v - ___(4)___;
    end
    % Normalize v by its Euclidean norm.
    Q(:, i) = v / ___(5)___;
end
    \end{verbatim}

    \begin{itemize}
        \item[(a)]
            (1) \verb"def", (2) \verb"A(i, :)", (3) \verb"Q(:, j)", (4) \verb"(q' * A(:, i)) * q", (5) \verb"norm(v)"
        \item[(b)]
            (1) \verb"function", (2) \verb"A(:, i)", (3) \verb"Q(:, i)", (4) \verb"(q' * v) * q", (5) \verb"norm(v)"
        \item[(c)]
            (1) \verb"def", (2) \verb"A(:, i)", (3) \verb"Q(:, i)", (4) \verb"(q' * v) * q", (5) \verb"sqrt(sum(v^2))"
        \item[(d)]
            (1) \verb"function", (2) \verb"A(:, i)", (3) \verb"Q(:, j)", (4) \verb"(q' * A(:, i)) * q", (5) \verb"norm(v)"
        \item[(e)]
            (1) \verb"function", (2) \verb"A(:, i)", (3) \verb"Q(:, j)", (4) \verb"(q' * v) * q", (5) \verb"sqrt(sum(v^2))"
        \end{itemize}
        
    \end{enumerate}
    
\end{enumerate}

\newpage

\noindent \textit{Solution.}
\begin{enumerate}
    \item[(1)] (a) True, (b) True, (c) True, (d) False, (e) True
    
    \vspace{2mm}
    
    An explain for (d):\\
    \verb"A = reshape(1:56, 7, 8);"\\
    \verb"A = A(14 <= A) .* A(43 >= A);"
    
    \vspace{2mm}
    
    Here, \verb"A(14 <= A)" and \verb"A(43 >= A)" are just logical statments. 
    Thus the \verb"A" gernerated in (d) consists of only 0s and 1s. 
    To achieve the goal, the task of selecting the correct values of \verb"A"
    using the generated logical variables should be added.
    
    \item[(2)] answer : (d)
    \begin{enumerate}
        \item[(1)] \verb"function"
        \item[(2)] \verb"A(:, i)"
        \item[(3)] \verb"Q(:, j)"
        \item[(4)] \verb"(q' * A(:, i)) * q"
        \item[(5)] \verb"norm(v)"
    \end{enumerate}
\end{enumerate}

\end{document}

