\documentclass[a4paper,10pt]{article}
\usepackage{amsthm}
\usepackage{amssymb}
\usepackage{color}
\usepackage{comment}
\usepackage{colonequals}
\usepackage{amsmath}
\usepackage{mleftright}
\usepackage{booktabs}
%\usepackage{mathtool}
\makeatletter

\NeedsTeXFormat{LaTeX2e}
\newif\if@widepage \@widepagetrue
\newif\if@solin \@solintrue
\newif\if@english \@englishfalse
\newif\if@solwithprob \@solwithprobtrue

\DeclareOption{nowidepage}{\@widepagefalse}
\DeclareOption{widepage}{\@widepagetrue}
\DeclareOption{solwithprob}{\@solwithprobtrue}
\DeclareOption{onlysol}{\@solwithprobfalse}
\DeclareOption{solafter}{\@solinfalse}
\DeclareOption{solin}{\@solintrue}
\DeclareOption{english}{\@englishtrue}
\DeclareOption{hangul}{\@englishfalse} \ProcessOptions

\usepackage{amsmath}
\usepackage{enumerate}

\newcommand{\highlight}[2]{\begingroup\setlength\fboxsep{1pt}\colorbox{#1}{#2}\endgroup}
\newcommand{\coloring}[2]{\begingroup\color{#1} #2\endgroup}
\definecolor{PTColor}{rgb}{0,0,1}%
\definecolor{CommentColor}{rgb}{1,0,0}%
\newcommand{\pt}{{\textbf{\coloring{PTColor}{(1 point)}}}}
\newcommand{\dpt}{{\textbf{\coloring{CommentColor}{($-$1 point)}}}}
\newcommand{\nPts}[1]{{\textbf{\coloring{PTColor}{(#1 points)}}}}
\newcommand{\dPts}[1]{{\textbf{\coloring{CommentColor}{($-$#1 points)}}}}

\newcommand{\tuborg}{\left\{\begin{array}{ll}}
\newcommand{\sluttuborg}{\end{array}\right.}

%\input epsf.sty
\def\baselinestretch{1.3}

\if@widepage \input{a4wide.sty}\fi
\newlength{\Problen}
\newlength{\Probheadlen}
\newlength{\Probskip}

\Probheadlen 16mm \Probskip 3mm \Problen\textwidth
\addtolength\Problen{-\Probheadlen} \addtolength\Problen{-\Probskip}

%%% \newcounter{Prob} %%%%%%%
\newsavebox{\@syntest}
\newbox\@lastproblem

\if@english
    \newenvironment{Prob}[2][]{ %
    \global\setbox\@lastproblem=\vbox\bgroup
    \par\noindent\parbox[t]{\Probheadlen}{\raggedleft\fontsize{10}{10pt}\selectfont
    {\bfseries\large #2 }\\[-1.5mm]
    \rule[1mm]{\Probheadlen}{.2mm}\\[-2mm]{\sffamily #1}}\hspace\Probskip%
    \begin{minipage}[t]{\Problen}}{\end{minipage}\par\vspace{3mm}\egroup%
    \unvcopy\@lastproblem}
    \newenvironment{ProbK}{\begin{lrbox}{\@syntest}%
    \begin{minipage}\textwidth}{\end{minipage}\end{lrbox}}
\else
    \newenvironment{ProbK}[2][]{ %
    \global\setbox\@lastproblem=\vbox\bgroup
    \par\noindent\parbox[t]{\Probheadlen}{\raggedleft\fontsize{10}{10pt}\selectfont
    {\bfseries\large #2 }\\[-1.5mm]
    \rule[1mm]{16mm}{.2mm}\\[-2mm]{\sffamily #1}}\hspace{\Probskip}%
    \begin{minipage}[t]{\Problen}}{\end{minipage}\par\vspace{3mm}\egroup%
    \unvcopy\@lastproblem}
    \newenvironment{Prob}{\begin{lrbox}{\@syntest}%
    \begin{minipage}\textwidth}{\end{minipage}\end{lrbox}}
\fi


\def\lhead{\@ifnextchar[{\@xlhead}{\@ylhead}}
\def\@xlhead[#1]#2{\gdef\@elhead{#1}\gdef\@olhead{#2}}
\def\@ylhead#1{\gdef\@elhead{#1}\gdef\@olhead{#1}}

\def\chead{\@ifnextchar[{\@xchead}{\@ychead}}
\def\@xchead[#1]#2{\gdef\@echead{#1}\gdef\@ochead{#2}}
\def\@ychead#1{\gdef\@echead{#1}\gdef\@ochead{#1}}

\def\rhead{\@ifnextchar[{\@xrhead}{\@yrhead}}
\def\@xrhead[#1]#2{\gdef\@erhead{#1}\gdef\@orhead{#2}}
\def\@yrhead#1{\gdef\@erhead{#1}\gdef\@orhead{#1}}

\def\lfoot{\@ifnextchar[{\@xlfoot}{\@ylfoot}}
\def\@xlfoot[#1]#2{\gdef\@elfoot{#1}\gdef\@olfoot{#2}}
\def\@ylfoot#1{\gdef\@elfoot{#1}\gdef\@olfoot{#1}}

\def\cfoot{\@ifnextchar[{\@xcfoot}{\@ycfoot}}
\def\@xcfoot[#1]#2{\gdef\@ecfoot{#1}\gdef\@ocfoot{#2}}
\def\@ycfoot#1{\gdef\@ecfoot{#1}\gdef\@ocfoot{#1}}

\def\rfoot{\@ifnextchar[{\@xrfoot}{\@yrfoot}}
\def\@xrfoot[#1]#2{\gdef\@erfoot{#1}\gdef\@orfoot{#2}}
\def\@yrfoot#1{\gdef\@erfoot{#1}\gdef\@orfoot{#1}}

\newdimen\headrulewidth
\newdimen\footrulewidth
\newdimen\plainheadrulewidth
\newdimen\plainfootrulewidth
\newdimen\headwidth
\newif\if@fancyplain \@fancyplainfalse
\def\fancyplain#1#2{\if@fancyplain#1\else#2\fi}

% Initialization of the head and foot text.

\headrulewidth 0.4pt \footrulewidth\z@ \plainheadrulewidth\z@
\plainfootrulewidth\z@

% Put together a header or footer given the left, center and
% right text, fillers at left and right and a rule.
% The \lap commands put the text into an hbox of zero size,
% so overlapping text does not generate an errormessage.

\def\@fancyhead#1#2#3#4#5{#1\hbox to\headwidth{\vbox{\hbox
{\rlap{\parbox[b]{\headwidth}{\raggedright#2\strut}}\hfill
\parbox[b]{\headwidth}{\centering#3\strut}\hfill
\llap{\parbox[b]{\headwidth}{\raggedleft#4\strut}}}\headrule}}#5}


\def\@fancyfoot#1#2#3#4#5{#1\hbox to\headwidth{\vbox{\footrule
\hbox{\rlap{\parbox[t]{\headwidth}{\raggedright#2\strut}}\hfill
\parbox[t]{\headwidth}{\centering#3\strut}\hfill
\llap{\parbox[t]{\headwidth}{\raggedleft#4\strut}}}}}#5}

\def\headrule{{\if@fancyplain\headrulewidth\plainheadrulewidth\fi
\hrule\@height\headrulewidth\@width\headwidth
\vskip-\headrulewidth}}

\def\footrule{{\if@fancyplain\footrulewidth\plainfootrulewidth\fi
\vskip-0.3\normalbaselineskip\vskip-\footrulewidth
\hrule\@width\headwidth\@height\footrulewidth\vskip0.3\normalbaselineskip}}

\def\ps@quiz{
\let\@mkboth\markboth
\@ifundefined{chapter}{\def\sectionmark##1{\markboth
{\uppercase{\ifnum \c@secnumdepth>\z@
 \thesection\hskip 1em\relax \fi ##1}}{}}
\def\subsectionmark##1{\markright {\ifnum \c@secnumdepth >\@ne
 \thesubsection\hskip 1em\relax \fi ##1}}}
{\def\chaptermark##1{\markboth {\uppercase{\ifnum
\c@secnumdepth>\m@ne
 \@chapapp\ \thechapter. \ \fi ##1}}{}}
\def\sectionmark##1{\markright{\uppercase{\ifnum \c@secnumdepth >\z@
 \thesection. \ \fi ##1}}}}
\lfoot{} \rfoot{\scriptsize Typeset by \LaTeX} \cfoot{}
\def\@oddhead{\@fancyhead\relax\@olhead\@ochead\@orhead\hss}
\def\@oddfoot{\@fancyfoot\relax\@olfoot\@ocfoot\@orfoot\hss}
\def\@evenhead{\@fancyhead\hss\@elhead\@echead\@erhead\relax}
\def\@evenfoot{\@fancyfoot\hss\@elfoot\@ecfoot\@erfoot\relax}
\headwidth\textwidth}

\pagestyle{quiz}

\newcommand{\ta}[5] {\lhead{\textsf{\bfseries #1  }}%
\rhead{\bfseries\sffamily  } %%%%%%%%%% Type your name, please!!
\chead{\bfseries\sffamily#2}
\lfoot{\scriptsize #4} \cfoot{\scriptsize page \thepage~of 6}}

\makeatother

\theoremstyle{remark}
\newtheorem*{sol}{Solution}

\newcommand{\emptybox}[2][\textwidth]{%
  \begingroup
  \setlength{\fboxsep}{-\fboxrule}%
  \noindent\framebox[#1]{\rule{0pt}{#2}}%
  \endgroup
}
%------------------------------------------------------
\ta{Spring 2021 MAS109 A-J}{Final Exam}{Final Exam}{}{10}
%------------------------------------------------------

\begin{document}

%%%You need to input the information of the problem as follows:
%%%%begin{ProbK}[Total points]{Problem Number}!!!!!!!!!!!!!


\begin{center}\textbf{Direction}\end{center}

\begin{enumerate}
\item The exam is for 65 minutes.
\item The total score is 90. There are 8 problems.
\end{enumerate}

\vfill \eject





\begin{ProbK}[pt 10]{1}
Choose all correct statements.

\begin{enumerate}
    \item[(a)] If $A^TA$ and $AA^T$ are both invertible then $A$ is square.
    \item[(b)] If $P$ is the standard matrix for the orthogonal projection of $\mathbb{R}^n$ onto a two dimensional subspace $W$, then the rank of $I-P$ is two.
    \item[(c)] If $A$ is a $3\times 3$ matrix with two pivot positions, then the equation $Ax=0$ has a nontrivial solution.
    \item[(d)] If $\mathrm{null}(A)=\{0\}$, then $A$ is invertible.
    \item[(e)] If $A$ is a $3\times 4$ matrix with $\mathrm{nullity}(A)=1$, the equation $Ax= \begin{bmatrix}
		1 \\
		2 \\
		0 \\
	\end{bmatrix}$ has infinitely many solutions.
\end{enumerate}
\end{ProbK}

\vskip 2cm 

\begin{ProbK}[pt 10]{2}
Choose all correct statements.

\begin{enumerate}
    \item[(a)] If $A$ is diagonalizable, then the characteristic polynomial of $A$ has $n$ distinct roots.
    \item[(b)] If $A$ is diagonalizable, then so is $A^k$ for every positive integer $k$.   
    \item[(c)] If the sum of the algebraic multiplicities of eigenvalues of $A$ is $n$, then $A$ is diagonalizable.
    \item[(d)] If $A$ is diagonalizable, then $A$ is orthogonally diagonalizable. 
    \item[(e)] Let $A$ be a symmetric and invertible matrix. If ${\bf x}^TA{\bf x}$ is a negative definite quadratic form, then so is ${\bf x}^TA^{-1}{\bf x}$. 
\end{enumerate}
\end{ProbK}


\vfill \eject


\begin{ProbK}[pt 10]{3}
Choose all correct statements.
\begin{enumerate}
	\item[(a)] If $A$ is diagonalizable, then there is a unique matrix $P$ such that $P^{-1}AP$ is a a diagonal matrix.
	\item[(b)] If an invertible matrix $A$ is diagonalizable, then $A^{-1}$ is also diagonalizable.
	\item[(c)] If $A$ is a square matrix, then $AA^T$ is orthogonally diagonalizable.
	\item[(d)] Let $A$ be a symmetric matrix. If there is a square matrix $C$ such that $A=C^TC$, then $A$ is positive definite.
	\item[(e)] Let $U\Sigma V^T$ be the singular value decomposition of an $m\times n$ matrix $A$. Then we can find orthonormal bases of the fundamental spaces of $A$. 
    \end{enumerate}


\end{ProbK}

\vskip2cm

\begin{ProbK}[pt 10]{4}
For positive integers $m,n$, let $M_{m\times n}(\mathbb{R})$ be the set of matrices of size $m\times n$ with entries in $\mathbb{R}$. Choose all correct statements.
\begin{enumerate}
	\item[(a)] Given a matrix $A\in M_{m\times n}(\mathbb{R})$. If both $A^TA$ and $AA^T$ are invertible, then $m=n$.
	\item[(b)] Given a matrix $A\in M_{m\times n}(\mathbb{R})$. If $\mathrm{rank}(A)=\mathrm{rank}(A^T)$, then $m=n$.
	\item[(c)] Given a matrix $A\in M_{m\times n}(\mathbb{R})$. The matrix $A$ and $AA^T$ have the same column space.
	\item[(d)] There exists a matrix $A\in M_{5\times 3}(\mathbb{R})$ such that $\mathrm{tr}(A^TA)=-2021$.
	\item[(e)] Given a matrix $A\in M_{3\times 5}(\mathbb{R})$, the column vectors of $A$ are linearly dependent.
    \end{enumerate}


\end{ProbK}

\vskip2cm

\begin{ProbK}[pt 10]{5}
Let $A\in M_{8\times 9}(\mathbb{R})$ with $\mathrm{rank}(A)=6.$ Choose all true statements.
\begin{enumerate}
	\item[(a)] $\dim(\mathrm{col}(A))=6.$
	\item[(b)] $\mathrm{nullity}(A)=3$.
	\item[(c)] $\dim(\mathrm{col}(A)^{\perp})=3$.
	\item[(d)] $\mathrm{nullity}(A^T)=3$.
	\item[(e)] $\dim(\mathrm{null}(A^T)^\perp)=6$.
\end{enumerate}

\end{ProbK}

\vfill \eject

\begin{ProbK}[pt 10]{6}
Let $\mathcal{B} = \{v_1, v_2, v_3\}$ and $\mathcal{C} = \{w_1, w_2, w_3\}$ be 
two bases of $\mathbb{R}^3$, where 
\begin{alignat*}{3}
	v_1 &= (1, 0, 1),\quad v_2 &&= (0,1,1),\quad v_3 &&=(1,1,0),\\
	w_1 &= (1,0,0),\quad w_2 &&= (1,1,0),\quad w_3 &&=(1,1,1).
\end{alignat*}
Let $\mathbf{x}$ be a vector in $\mathbb{R}^3$ whose coordinate with respect to $\mathcal{B}$ is 
$[\mathbf{x}]_\mathcal{B} = (1, 2, 3)$. If $[\mathbf{x}]_\mathcal{C} = (a, b, c)$, which of the following is $a+b+c$?
\begin{enumerate}
	\item[(a)] 0
	\item[(b)] 1
	\item[(c)] 2
	\item[(d)] 3
	\item[(e)] 4
\end{enumerate}
\end{ProbK}

\vskip2cm

\begin{ProbK}[pt 10]{7}
Find $a+b$ where $y = ax+b$ is the least square fit line to the following four points: 
(2, 1), (3, 1), (5, 3), (6, 4).
    \begin{enumerate}
    \item[(a)] $\frac{1}{8}$
    \item[(b)] $\frac{2}{9}$
    \item[(c)] $\frac{3}{20}$
    \item[(d)] $\frac{7}{40}$
    \item[(e)] $\frac{1}{4}$
    \end{enumerate}
\end{ProbK}

\vfill \eject

\begin{ProbK}[pt 10]{8}
\begin{enumerate}
	\item[(a)] (10 points, 2 points for each)
	
	\vspace{1mm}
	Let $A$ be a $7 \times 8$ matrix given by
	\begin{align*}
		A =
		\begin{bmatrix}
			1 &  8 & 0 & 0 & 0 & 0 & 0 & 50\\
			2 &  9 & 0 & 0 & 0 & 0 & 44 & 51\\
			3 & 10 & 0 & 0 & 0 & 0 & 45 & 52\\
			4 & 11 & 0 & 0 & 0 & 0 & 46 & 53\\
			5 & 12 & 0 & 0 & 0 & 0 & 47 & 54\\
			6 & 13 & 0 & 0 & 0 & 0 & 48 & 55\\
			7 & 0 & 0 & 0 & 0 & 0 & 49 & 56\\
		\end{bmatrix}.
	\end{align*}
	
	Write \textbf{True} for codes that generate the same matrix $A$ as above, or \textbf{False} for codes that do not.
	
	\vspace{2mm}
	
	\begin{itemize}
		\item[(a)] 
		\verb"A = reshape(1:56, 7, 8);"\\
		\verb"A(14 <= A & 43 >= A) = 0;"
		
		\item[(b)] 
		\verb"A = reshape(1:56, 7, 8);"\\
		\verb"A = A .* ((14 > A) + (43 < A));"
		
		\item[(c)]
		\verb"A(1:13) = 1:13;   A(44:56) = 44:56;"\\
		\verb"A = reshape(A, 7, 8);"
		
		\item[(d)] 
		\verb"A = reshape(1:56, 7, 8);"\\
		\verb"A = A(14 <= A) .* A(43 >= A);"
		
		\item[(e)] 
		\verb"A = zeros(7, 8);"\\
		\verb"A(:) = [1:13, zeros(1, 30), 44:56];"
	\end{itemize}
\end{enumerate}
\end{ProbK}

\vfill \eject

\begin{ProbK}[pt 10]{8}
	\begin{enumerate}
	\item[(b)] (10 potins)
	
	\vspace{1mm}
	
	Let $A$ be an $m \times n$ matrix with full column rank. 
	The following MATLAB function performs a Gram-Schmidt process and produces 
	an $m \times n$ orthogonal matrix $Q$ whose columns form an orthonormal basis for col$(A)$. 
	Fill in the blanks (1) - (5).\\
	(In this problem, assume that the input $A$ is a matrix 
	that gives the same result for the both of classical Gram-Schmidt
	and the modified Gram-Schmidt.)
	
	\vspace{2mm}
	
	\begin{verbatim}
		% ------- The following is the script file `GramSchmidt.m'. -------
		___(1)___ Q = GramSchmidt(A)
		[m, n] = size(A);
		
		% Initialize the matrix Q as an m*n zero matrix.
		Q = zeros(m, n);
		for i = 1 : n
		% v starts with a column of A.
		v = ___(2)___;
		for j = 1 : i-1
		% Subtract orthogonal projections of v onto the subspaces
		% spanned by the previously generated orthonormal vectors.
		q = ___(3)___;
		v = v - ___(4)___;
		end
		% Normalize v by its Euclidean norm.
		Q(:, i) = v / ___(5)___;
		end
	\end{verbatim}
	
	\begin{itemize}
		\item[(a)]
		(1) \verb"def", (2) \verb"A(i, :)", (3) \verb"Q(:, j)", (4) \verb"(q' * A(:, i)) * q", (5) \verb"norm(v)"
		\item[(b)]
		(1) \verb"function", (2) \verb"A(:, i)", (3) \verb"Q(:, i)", (4) \verb"(q' * v) * q", (5) \verb"norm(v)"
		\item[(c)]
		(1) \verb"def", (2) \verb"A(:, i)", (3) \verb"Q(:, i)", (4) \verb"(q' * v) * q", (5) \verb"sqrt(sum(v^2))"
		\item[(d)]
		(1) \verb"function", (2) \verb"A(:, i)", (3) \verb"Q(:, j)", (4) \verb"(q' * A(:, i)) * q", (5) \verb"norm(v)"
		\item[(e)]
		(1) \verb"function", (2) \verb"A(:, i)", (3) \verb"Q(:, j)", (4) \verb"(q' * v) * q", (5) \verb"sqrt(sum(v^2))"
	\end{itemize}
\end{enumerate}
\end{ProbK}

\vfill \eject
\end{document}