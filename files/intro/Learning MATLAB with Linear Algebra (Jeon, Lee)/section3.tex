% section3
\chapter {Matrices and Matrix Algebra}

\section{Operations on Matrices}

No MATLAB problems in this section.


\section{Inverses; Algebraic Properties of Matrices}

\begin{exer}
In this problem, we compute $A^{5} - 3A^{3} + 7A - 4I$ for the matrix $A$, where
$$
A = \left[\begin{array}{rrrr} 1&\hspace{3mm} 2&\hspace{1mm} -3&\hspace{2mm} 0\\ 1 & 1 & -2 & 1 \\ 2 & 1 & 3 & 4 \\ -3 & 2 & 2 & -8 \end{array} \right].
$$
\begin{enumerate}

\vspace{2mm}
\item[(a)]  Using the syntax $A$\hspace{1mm}$\hat{}$\hspace{1mm}$k$ which produces the $k$-th power of a square matrix and the command \textit{eye} for the identity matrix, compute the above matrix polynomial.

\vspace{1mm}
\item[(b)]  Using the command \textit{polyvalm}, compute the above matrix polynomial.

\vspace{1mm}
\item[(c)]  Tell what happens if you type the syntax $A.$\hspace{1mm}$\hat{}$\hspace{1mm}$k$.

\end{enumerate}
\end{exer}

\begin{sol}

\begin{verbatim}

% Construct the matrix A.
A = [1 2 -3 0; 1 1 -2 1; 2 1 3 4; -3 2 2 -8]; 

% (a)
result_a = A^5 + (-3)*A^3 + 7*A + (-4)*eye(4);

% Display the matrix polynomial.
disp('The result of the matrix polynomial is');
disp(result_a) 

% (b)
% Coefficient of the matrix polynomial.
coeff_poly = [1 0 -3 0 7 -4]; 

% Evaluate the matrix polynomial of coefficient
% with coeff_poly vector with the input matrix A.
result_b = polyvalm(coeff_poly, A);

% Display the matrix polynomial.
disp('The result of the matrix polynomial is');
disp(result_b);

% (c)
disp('The result of A.^2 is'); disp(A.^2);
disp('The result of A.^3 is'); disp(A.^3);
disp('The result of A.^4 is'); disp(A.^4);
\end{verbatim}

\begin{outputs}

\begin{verbatim}

The result of the matrix polynomial is
         874       -1272         -39        3021
        2580       -2306        -723        7536
        5191       -4121       -2444       14563
      -16852       12539        5649      -46917

The result of the matrix polynomial is
         874       -1272         -39        3021
        2580       -2306        -723        7536
        5191       -4121       -2444       14563
      -16852       12539        5649      -46917

The result of A.^2 is
     1     4     9     0
     1     1     4     1
     4     1     9    16
     9     4     4    64

The result of A.^3 is
     1     8   -27     0
     1     1    -8     1
     8     1    27    64
   -27     8     8  -512

The result of A.^4 is
           1          16          81           0
           1           1          16           1
          16           1          81         256
          81          16          16        4096
\end{verbatim}
\end{outputs}

\noindent From the results, we can see that the syntax $A.$\hspace{1mm}$\hat{}$\hspace{1mm}$k$ produces the entrywise $k$-th powers of the matrix $A$.

\end{sol}



\section{Elementary Matrices; A Method for Finding $A^{-1}$}



\begin{exer}

In this problem, we solve the linear system $A \mathbf{x} = \mathbf{b}$ by using matrix inversion, where
$$
A = \left[\begin{array}{rrrr} 3 &\hspace{2.5mm} 3 & -4 & -3 \\ 0 & 6 & 1 & 1\\ 5 & 4 & 2 & 1 \\ 2 & 3 & 3 & 2 \end{array} \right] \hspace{2mm} \mathrm{and} \hspace{3mm} \textbf{b} = \left[\begin{array}{r} -2 \\ 3 \\ 5 \\ 1 \end{array} \right].
$$

\begin{enumerate}
%8a
\item[(a)] Use the MATLAB command \textit{inv} or the syntax $A$\hspace{1mm}$\hat{}$\hspace{1mm}$(-1)$ to find the inverse of $A$.
\vspace{1mm}
%8b
\item[(b)] Display the output matrix as a rational form, NOT decimally. You may use the command \textit{format}.
\vspace{1mm}
%8c
\item[(c)] Using the result of (a), compute the solution of the linear system $A \mathbf{x} = \mathbf{b}$ by taking $\mathbf{x} = A^{-1} \mathbf{b}$.

\end{enumerate}

\end{exer}


\begin{sol}

\begin{verbatim}

% Construct the matrix A and the right-hand-side vector b.
A = [3 3 -4 -3; 0 6 1 1; 5 4 2 1; 2 3 3 2]; 
b = [-2 3 5 1]'; 

% (a)
% Use the command inv.
Inv_A1 = inv(A); 

% Use the syntax A^(-1).
Inv_A2 = A^(-1); 

% (b)
format rat; 
disp('The result of the command inv is'); disp(Inv_A1);
disp('The result of the syntax A^(-1) is'); disp(Inv_A2);

% (c)
% Since A is invertible, the solution to Ax=b is x=A^(-1)*b.
x = Inv_A1 * b;
disp('The solution to Ax=b is x = A^(-1)*b'); disp(x');
\end{verbatim}

\begin{outputs}

\begin{verbatim}

The result of the command inv is
 -7    5   12  -19
  3   -2   -5    8
 41  -30  -69  111
-59   43   99 -159

The result of the syntax A^(-1) is
 -7    5   12  -19
  3   -2   -5    8
 41  -30  -69  111
-59   43   99 -159

The solution to Ax=b is x = A^(-1)*b
    70      -29     -406      583
\end{verbatim}
\end{outputs}
\end{sol}



\section{Subspaces and Linear Independence}


\begin{exer} (\textit{Sigma notation})\\
Compute the linear combination 
$$\mathbf{v}=\Sigma_{j=1}^{25} c_{j}\mathbf{v}_{j}$$
for $c_{j}=1/j$ and $\mathbf{v}_{j}=(\sin j, \cos j).$

\end{exer}



\begin{sol}
\begin{verbatim}

v=zeros(1,2);
for i=1:25
    v=v+(1/i)*[sin(i), cos(i)];
end
disp(v);
\end{verbatim}



\begin{outputs}
\begin{verbatim}

1.0322    0.0553
\end{verbatim}
\end{outputs}
\end{sol}

\vspace{3mm}


\begin{exer} Let $\mathbf{v_{1}}=(4, 3, 2, 1)$, $\mathbf{v_{2}}=(5, 1, 2, 4)$, $\mathbf{v_{3}}=(7, 1, 5, 3)$, $\mathbf{x}=(16, 5, 9, 8)$, and $\mathbf{y}=(3, 1, 2, 7)$. Determine whether $\mathbf{x}$ and $\mathbf{y}$ lie in $\textrm{span}\{\mathbf{v_{1}}, \mathbf{v_{2}}, \mathbf{v_{3}}\}$.

\end{exer}


\begin{sol}

\begin{verbatim}

% Construct v1, v2, v3, x, y
v1=[4 3 2 1]'; v2=[5 1 2 4]'; v3=[7 1 5 3]';
x=[16 5 9 8]'; y=[3 1 2 7]';

% Augmented matrices [v1|v2|v3|x] and [v1|v2|v3|y]
X=[v1 v2 v3 x];
Y=[v1 v2 v3 y];

disp('Reduced row echelon form of [v1 v2 v3 x] is');
disp(rref(X));
disp('Reduced row echelon form of [v1 v2 v3 y] is');
disp(rref(Y));
\end{verbatim}


\begin{outputs}

\begin{verbatim}

Reduced row echelon form of [v1 v2 v3 x] is
       1              0              0              1       
       0              1              0              1       
       0              0              1              1       
       0              0              0              0       

Reduced row echelon form of [v1 v2 v3 y] is
       1              0              0              0       
       0              1              0              0       
       0              0              1              0       
       0              0              0              1      
\end{verbatim}
\end{outputs}

\noindent Therefore, $\mathbf{x}$ lies in $\textrm{span}\{\mathbf{v_{1}}, \mathbf{v_{2}}, \mathbf{v_{3}}\}$ and $\mathbf{y}$ does not lie in $\textrm{span}\{\mathbf{v_{1}}, \mathbf{v_{2}}, \mathbf{v_{3}}\}$. 
\end{sol}


%\begin{exer} (\textit{Linear Combinations})\\
%Use the MATLAB command \textit{rref} to express the vector $\mathbf{b}=(-21, \hspace{1mm}-60, \hspace{1mm}-3, \hspace{1mm}108, \hspace{1mm}84)$ as a linear combination of $\mathbf{v_{1}}$, $\mathbf{v_{2}}$, and $\mathbf{v_{3}}$ where $\mathbf{v_{1}}=(1, \hspace{1mm} -1, \hspace{1mm}3, \hspace{1mm}11, \hspace{1mm}20)$, $\mathbf{v_{2}}=(10, \hspace{1mm}5, \hspace{1mm}15, \hspace{1mm}20, \hspace{1mm}11)$, and $\mathbf{v_{3}}=(3, \hspace{1mm}3, \hspace{1mm}4, \hspace{1mm}4, \hspace{1mm}9)$.
%
%\end{exer}
%
%
%\begin{sol}
%
%\begin{verbatim}
%
%% Construct b as a column vector.
%b = [-21 -60 -3 108 84]'; 
%
%% Set v1, v2, v3 as column vectors.
%v1 = [1 -1 3 11 20]';
%v2 = [10 5 15 20 11]';
%v3 = [3 3 4 4 9]'; 
%
%% Set a matrix A with column vectors v1, v2 and v3. 
%A = [v1 v2 v3];
%
%% Construct the augmented matrix [A | b].
%augA = [A b]; 
%
%% Find the reduced row echelon form of augA.
%rref_augA = rref(augA); 
%
%% Extract the solution vector from rref_augA.
%x = rref_augA(1:3, 4); 
%
%% From the result of rref_augA, we get rank(A) = rank([A | b]),
%% hence, we can find each coefficient of this linear combination.
%% Otherwise, we cannot find a linear combination of b as v1, v2, and v3.
%
%% Moreover, since rank(A) = rank([A | b]) = the number of columns of A,
%% b is uniquely expressed as a linear combination of v1, v2, and v3.
%
%format rat % Display the result as an integer form.
%disp('b is a linear combination of x(1)*v1+x(2)*v2+x(3)*v3, where');
%disp('x(1) ='); disp(x(1)); 
%disp('x(2) ='); disp(x(2));
%disp('x(3) ='); disp(x(3));
%\end{verbatim}
%
%
%\begin{outputs}
%
%\begin{verbatim}
%
%b is a linear combination of x(1)*v1+x(2)*v2+x(3)*v3, where
%x(1) =
%      12
%
%x(2) =
%       3
%
%x(3) =
%     -21
%\end{verbatim}
%\end{outputs}
%\end{sol}

\section{The Geometry of Linear Systems}


No MATLAB problems in this section.

%%% Week2
\newpage
\section{Matrices with Special Forms}

\begin{exer} (\textit{Inverting $(I-A)$})
\begin{enumerate}
%7a
\item[(a)]  (\textit{Inverting $(I-A)$ when $A$ is nilpotent}) Using MATLAB, show that the matrix 
$$
A = \left[\begin{array}{rrr} 2&\quad 11&\quad 3\\ -2 & -11 & -3\\ 8 & 35 & 9 \end{array} \right]
$$
is nilpotent, and then use Theorem~3.6.6 in the text book to compute $(I-A)^{-1}$. Check your answer by computing the inverse directly in MATLAB.
\vspace{1mm}

\item[(b)] (\textit{Approximating $(I-A)^{-1}$ by a power series}) Using MATLAB, confirm that the matrix
$$
A = \left[\displaystyle\begin{array}{rrr} 0&\quad \displaystyle\frac{1}{4}&\quad \displaystyle\frac{1}{8}\\ \displaystyle\frac{1}{4} & \displaystyle\frac{1}{8} & \displaystyle\frac{1}{10}\\ \displaystyle\frac{1}{8} & \displaystyle\frac{1}{10} & \displaystyle\frac{1}{10} \end{array} \right]
$$
satisfies the condition in Theorem~3.6.7 of the text book. You may use the command \textit{sum}. Since $A$ satisfies that condition, $(I-A)$ is invertible and can be expressed by the series in Formula~(18) in Section~3.6 of the text book. Compute the approximation $$(I-A)^{-1}\approx I+A+A^2+A^3+\cdots+A^{10},$$ and compare it with the inverse of $I-A$ produced directly by MATLAB. To how many decimal places do the results agree? You may use the command \textit{format} to display the output with long digits.

\end{enumerate}
\end{exer}


\begin{sol}
\verb""
\begin{enumerate}
\item[(a)]
\begin{verbatim}
% (a)-i
A = [ 2 11 3 ; -2 -11 -3; 8 35 9];  % Construct the matrix A.
% Compute the A^2, A^3, ... , and display.
disp('A^2 is'); disp(A^2);
disp('A^3 is'); disp(A^3);

% (a)-ii Comparing two result

% By Theorem 3.6.6, (I-A)^(-1)=I+A+A^2.
result1=eye(3)+A+A^2;  

% Compute the inverse of (I-A) directly.
result2=inv(eye(3)-A);
disp('I+A+A^2 is'); disp(result1);
disp('(I-A)^(-1) is'); disp(result2);

% Display as a rational form.
format rat;	
disp('Rational form of (I-A)^(-1) is');disp(result2);
\end{verbatim}

\begin{outputs}

\begin{verbatim}

A^2 is
     6     6     0
    -6    -6     0
    18    18     0

A^3 is
     0     0     0
     0     0     0
     0     0     0

I+A+A^2 is
     9    17     3
    -8   -16    -3
    26    53    10

(I-A)^(-1) is
    9.0000   17.0000    3.0000
   -8.0000  -16.0000   -3.0000
   26.0000   53.0000   10.0000

Rational form of (I-A)^(-1) is
       9             17              3       
      -8            -16             -3       
      26             53             10       
\end{verbatim}

\end{outputs}

\noindent Since $A^{3} = \mathbf{0}$, $A$ is nilpotent. 
By the Theorem $3.6.6$, since $A^{3} = \mathbf{0}$, $I-A$ is invertible and $(I-A)^{-1} = I + A + A^{2}.$ To check answer by computing the inverse directly in MATLAB, we implement as in the next page.


\item[(b)]
\begin{verbatim}
% Construct the matrix A.
A=[0 1/4 1/8; 1/4 1/8 1/10; 1/8 1/10 1/10]; 

% Check that the condition in Theorem 3.6.7 
% of the text book is satisfied for matrix A.
column_sum=sum(abs(A),1);   % column-wise sum 
row_sum=sum(abs(A),2);  % row-wise sum
disp('The sum of the absolute values of the entries in each column is');
disp(column_sum);
disp('The sum of the absolute values of the entries in each row is');
disp(row_sum);

result3=eye(size(A))+A+A^2+A^3+A^4+A^5+A^6+A^7+A^8+A^9+A^10;
result4=inv(eye(3)-A);

format long;	% Display the result with long digits
disp('With format long');
disp('Approximated inv(I-A) is'); disp(result3);
disp('Exact inv(I-A) is'); disp(result4);
\end{verbatim}


\begin{outputs}

\begin{verbatim}

The sum of the absolute values of the entries in each column is
       3/8           19/40          13/40    

The sum of the absolute values of the entries in each row is
       3/8     
      19/40    
      13/40    

With format long
Approximated inv(I-A) is
   1.108587459181130   0.338615080927493   0.191581699462210
   0.338615080927493   1.260966638806045   0.187122081247432
   0.191581699462210   0.187122081247432   1.158500720998029

Exact inv(I-A) is
   1.108610894508188   0.338643199287067   0.191600757491367
   0.338643199287067   1.261000334187368   0.187144925921800
   0.191600757491367   0.187144925921800   1.158516208087334
\end{verbatim}

\end{outputs}


\noindent The approximation result agrees with the exact result to 2 decimal places.
\end{enumerate}
\end{sol}


\section{Matrix Factorizations; $LU$-Decomposition}

\begin{exer}(\textit{LU-decompositions})
In this problem, we find an $LU$-decomposition of $A$, where $A$ is given in the \mbox{Example $2$} of the Section $3.7$.

\vspace{2mm}
\begin{enumerate}
\item[(a)] Find an $LU$-decomposition of $A$ by following the procedure given in the Example $2$.
\vspace{1mm}
\item[(b)] Solve the linear system $A \mathbf{x} = \mathbf{b}$ by using the $LU$-decomposition of $A$ obtained in (a), where $\textbf{b} = \left[\begin{array}{r} 0 \\ -2 \\ 1 \end{array} \right].$
\vspace{1mm}
\item[(c)] Tell what happens if you use the MATLAB command \textit{lu} of $A$. Explain why this result differs from the result in (a).

\end{enumerate}

\end{exer}


\begin{sol}

\begin{verbatim}

%(a)
A = [6 -2 0; 9 -1 1; 3 7 5]; % Set the matrix A.

format rat; % Display results as a rational form.

% Initialization of U and L.
U = A; L = eye(3); 

% Multiply the first row by 1/6.
U(1,:)=(1/6)*U(1,:); 
% L(1,1) is the inverse of 1/6.
L(1,1)=(1/6)^(-1); 

% Add (-9) times the first to the second.
U(2,:)=((-9)*U(1,:))+U(2,:);
% L(2,1) is the negative of (-9).
L(2,1)=-(-9); 

% Add (-3) times the first to the third.
U(3,:)=((-3)*U(1,:))+U(3,:); 
% L(3,1) is the negative of (-3).
L(3,1)=-(-3); 

% Multiply the second row by 1/2.
U(2,:)=(1/2)*U(2,:);
% L(2,2) is the inverse of 1/2.
L(2,2)=(1/2)^(-1); 

% Add (-8) times the second to the third.
U(3,:)=((-8)*U(2,:))+U(3,:); 
% L(3,2) is the negative of (-8).
L(3,2)=-(-8); 

disp('A is'); disp(A);
disp('The Lower Triangular part L is'); disp(L);
disp('The Upper Triangular part U is'); disp(U);
disp('The product L*U is'); disp(L*U);

%(b)
% Solve the linear system Ax=b 
% by using the LU-decomposition obtained in (a).

% First, let us solve L*y = b by forward substitution.
 % Set the right-hand-side vector b.
b = [0 -2 1]';

 % Initialization of the solution vector y.
y = zeros(3, 1);
y(1) = b(1) / L(1, 1);
y(2) = (b(2) - (L(2, 1)*y(1))) / L(2, 2);
y(3) = (b(3) - (L(3, 1)*y(1)) - (L(3, 2)*y(2))) / L(3, 3);

% Next, let us solve U*x = y by backward substitution.
x = zeros(3, 1); % Initialization of the solution vector x.
x(3) = y(3) / U(3, 3);
x(2) = (y(2) - (U(2, 3)*x(3))) / U(2, 2);
x(1) = (y(1) - (U(1, 3)*x(3)) - (U(1, 2)*x(2))) / U(1, 1);

disp('The solution to Ax=b by the LU-decomposition is'); disp(x');

% (c)
fprintf('Using MATLAB command lu\n');
% LU decomposition of A with a permutation matrix.
[L U P] = lu(A);

disp('Lower triangular part L is'); disp(L);
disp('Upper triangular part U is'); disp(U);
disp('The permutation matrix P is'); disp(P);
disp('PA='); disp(P*A); disp('LU='); disp(L*U);

\end{verbatim}

\begin{outputs}

\begin{verbatim}

A is
       6             -2              0
       9             -1              1
       3              7              5

The Lower Triangular part L is
       6              0              0
       9              2              0
       3              8              1

The Upper Triangular part U is
       1             -1/3            0
       0              1              1/2
       0              0              1

The product L*U is
       6             -2              0
       9             -1              1
       3              7              5

The solution to Ax=b by the LU-decomposition is
     -11/6          -11/2            9

Using MATLAB command lu

Lower triangular part L is
       1              0              0
       1/3            1              0
       2/3           -2/11           1
Upper triangular part U is
       9             -1              1
       0             22/3           14/3
       0              0              2/11
The permutation matrix P is
       0              1              0
       0              0              1
       1              0              0
PA=
       9             -1              1
       3              7              5
       6             -2              0
LU=
       9             -1              1
       3              7              5
       6             -2              0

\end{verbatim}

\end{outputs}

\noindent Since the permutation matrix $P$ is not the identity matrix, the MATLAB command \textit{lu} gave us an $LU$-decomposition after multiplying $A$ by the permutation matrix $P$, hence, this decomposition is a $PLU$-decomposition of $A$ because $PA=LU$. Since at least one row interchange of $A$ occurred in the process of $LU$-decomposition, this result is different from the previous decomposition obtained in $(a)$.

\end{sol}


\vspace{3mm}
\begin{exer} (\textit{LU-decomposition})
\begin{enumerate}
\item[(a)] The MATLAB command \textit{lu} is used to find the $LU$-decomposition of a matrix $A$. Tell what happens if you use the command $lu$ for $A$, where $A$ is given in the Example 2 of the Section 3.7. Explain why this result differs from the result in the textbook.
\vspace{1mm}
\item[(b)] Using MATLAB, observe what happens when you try to find an $LU$-decomposition of a singular matrix.

\end{enumerate}
\end{exer}

\begin{sol}

\begin{verbatim}

% (a)
% Construct the matrix A.
A=[6 -2 0; 9 -1 1; 3 7 5]; 

% LU decomposition of A.
[L U P]=lu(A); 
disp('[L U P]=lu(A)');
disp('L'); disp(L); disp('U'); disp(U); disp('P'); disp(P);

% (b)
% Construct the some singular matrices.
A1=[1 0 0; -2 0 0; 4 6 1]; 
A2=[1 -2 7; -4 8 5; 2 -4 3];
A3=[1 0 0; -2 0 0; 4 6 1]; 

% LU decompositions of them.
[L1 U1 P1]=lu(A1); [L2 U2 P2]=lu(A2); [L3 U3 P3]=lu(A3); 
disp('[L1 U1 P1]=lu(A1)'); disp('L1');disp(L1);disp('U1');disp(U1);
disp('[L2 U2 P2]=lu(A2)'); disp('L2');disp(L2); disp('U2');disp(U2);
disp('[L3 U3 P3]=lu(A3)'); disp('L3');disp(L3); disp('U3');disp(U3);
\end{verbatim}

\begin{outputs}

\begin{verbatim}

[L U P]=lu(A)
L
    1.0000         0         0
    0.3333    1.0000         0
    0.6667   -0.1818    1.0000

U
    9.0000   -1.0000    1.0000
         0    7.3333    4.6667
         0         0    0.1818

P
     0     1     0
     0     0     1
     1     0     0


[L1 U1 P1]=lu(A1)
L1
    1.0000         0         0
   -0.5000    1.0000         0
    0.2500   -0.5000    1.0000
U1
    4.0000    6.0000    1.0000
         0    3.0000    0.5000
         0         0         0
         
[L2 U2 P2]=lu(A2)
L2
    1.0000         0         0
   -0.2500    1.0000         0
   -0.5000         0    1.0000
U2
   -4.0000    8.0000    5.0000
         0         0    8.2500
         0         0    5.5000
         
[L3 U3 P3]=lu(A3)
L3
    1.0000         0         0
   -0.5000    1.0000         0
    0.2500   -0.5000    1.0000
U3
    4.0000    6.0000    1.0000
         0    3.0000    0.5000
         0         0         0
         
\end{verbatim}
\end{outputs}

\noindent \textit{Remark on (a).} Since the permutation matrix $P$ is not the identity matrix, the MATLAB command \textit{lu} gave us an $LU$-decomposition after multiplying $A$ by the permutation matrix $P$, hence, this decomposition is a $PLU$-decomposition of $A$ because $PA=LU$. Since at least one row interchange of $A$ occurred in the process of $LU$-decomposition, this result is different from the decomposition result in the textbook.

\vspace{2mm}
\noindent \textit{Remark on (b).} When we try $LU$-decomposition of the sigular matrices using the MATLAB command \textit{lu}, the resulting upper triangular matrices are singular.

\end{sol}