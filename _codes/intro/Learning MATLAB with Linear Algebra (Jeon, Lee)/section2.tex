% section 2
\chapter {Systems of Linear Equations}

%\section{Introduction to Systems of Linear Equations}
%\section{Solving Linear Systems by Row Reduction}

\section{Introduction to Systems of Linear Equatios}

No MATLAB problems in this section.

\section{Solving Linear Ssytems by Row Reduction}
\begin{exer}
(\textit{Reduced Row Echelon Form with Pivot Columns and Ranks}) \\
In MATLAB, there are several useful commands for matrices such as \textit{rref} command which produces the reduced row echelon form together with the pivot columns, and \textit{rank} command which gives the number of the leading $1$'s without finding its row echelon form. Find the reduced row echelon form, the pivot columns, and the rank of the matrix $A$, where
\vspace{2mm}
\begin{displaymath}
A = \left[\begin{array}{rrrrr} 2& \hspace{1mm}-3& \hspace{1mm} 1& \hspace{3mm} 0& \hspace{2mm} 4 \\ 1 & 1 & 2 & 2 & 0 \\ 3 & 0 & -1 & 4 & 5 \\ 1 & 6 & 5 & 6 & -4 \end{array} \right].
\end{displaymath}
\end{exer}

\begin{sol}

\begin{verbatim}

% Construct the matrix A.
A=[2 -3 1 0 4; 1 1 2 2 0; 3 0 -1 4 5; 1 6 5 6 -4]; 

% Display the format of each entry as a rational form
format rat; 

% Find the reduced row echelon form 
% and the pivot columns of the matrix A.
[rref_A pivotcols] = rref(A);

% Find the rank of the matrix A.
rank_A = rank(A); 

disp('The reduced row echelon form is'); disp(rref_A);
disp('The pivot columns are'); disp(pivotcols);
disp('The number of the leading 1 is'); disp(rank_A);
\end{verbatim}

\begin{outputs}

\begin{verbatim}

The reduced row echelon form is
       1    0    0   17/13   3/2
       0    1    0   11/13  -1/2
       0    0    1   -1/13  -1/2
       0    0    0      0     0

The pivot columns are
       1    2    3

The number of leading 1 is
       3
\end{verbatim}
\end{outputs}
\end{sol}


\vspace{5mm}
\begin{exer}
(\textit{Linear Combinations}) Use the MATLAB command \textit{rref} to express the vector $\mathbf{b}=(-21, \hspace{1mm}-60, \hspace{1mm}-3, \hspace{1mm}108, \hspace{1mm}84)$ as a linear combination of $\mathbf{v_{1}}$, $\mathbf{v_{2}}$, and $\mathbf{v_{3}}$ where 
$\mathbf{v_{1}}=(1, \hspace{1mm} -1, \hspace{1mm}3, \hspace{1mm}11, \hspace{1mm}20)$, 
$\mathbf{v_{2}}=(10, \hspace{1mm}5, \hspace{1mm}15, \hspace{1mm}20, \hspace{1mm}11)$, 
and 
$\mathbf{v_{3}}=(3, \hspace{1mm}3, \hspace{1mm}4, \hspace{1mm}4, \hspace{1mm}9)$.
\end{exer}

\begin{sol}
\begin{verbatim}

% Construct b as a column vector.
b = [-21 -60 -3 108 84]';
% Set v1, v2, v3 as column vectors. 
v1 = [1 -1 3 11 20]'; 
v2 = [10 5 15 20 11]'; 
v3 = [3 3 4 4 9]';
% Set a matrix A with column vectors v1, v2 and v3. 
A = [v1 v2 v3]; 
% Augmented matrix [A | b].
augA = [A b]; 
% Reduced row echelon form of augA.
rref_augA = rref(augA);
% Solution vector from rref_augA. 
x = rref_augA(1:3, 4); 

% Display the result as an integer form.
format rat; 
disp('b is a linear combination of x(1)*v1+x(2)*v2+x(3)*v3, where');
disp('x(1) ='); disp(x(1)); disp('x(2) ='); disp(x(2));
disp('x(3) ='); disp(x(3));
\end{verbatim}

\begin{outputs}

\begin{verbatim}

b is a linear combination of x(1)*v1+x(2)*v2+x(3)*v3, where
x(1) =
      12

x(2) =
       3

x(3) =
     -21
\end{verbatim}
\end{outputs}
\end{sol}