% section6
\chapter{Linear Transformations}

\section{Matrices as Transformations}

\begin{exer} (\textit{Linear Transformation: Rotation})\\
Make a function file with a function name \textit{reflect\_pt} to find the reflection of the point $(a,b)$ about the line through the origin of the $xy$-plane that makes an angle of $\theta\,^{\circ}$ with the positive $x$-axis. Make $a$, $b$, and $\theta$ the inputs to the function and the reflection point $(x,y)$ the output. Using this function, find the result of this function for $(a,b)=(1,3)$ and $\theta=12$ by the following commands:

\vspace{2mm}

\begin{verbatim}
>> [x, y]=reflect_pt(1, 3, 12)
\end{verbatim}

\end{exer}



\begin{sol}
\begin{verbatim}

%--- The following is the function file 'reflect_pt.m'. ---%
function [x, y]=reflect_pt(a, b, ang)
  theta=ang*(pi/180);
  T = [cos(2*theta) sin(2*theta); sin(2*theta) -cos(2*theta)];
    
  result=T*[a b]';
  x=result(1); y=result(2);
end
\end{verbatim}

\begin{outputs}

\begin{verbatim}

x =
    2.1338
    
y =
   -2.3339
\end{verbatim}
\end{outputs}
\end{sol}

\vspace{3mm}

\begin{exer} (\textit{Linear Transformation: Projection})\\ 
Consider successive rotations of $\mathbb{R}^{3}$ by an angle $\theta_{1}$ degree about the $x$-axis, then by an angle $\theta_{2}$ degree about the $y$-axis, and then by an angle $\theta_{3}$ degree about the $z$-axis. Make a function file named \textit{comp\_Rot} to find an appropriate axis and angle of rotation that achieves the same result in one rotation. Make $\theta_{1}, \theta_{2},$ and $\theta_{3}$ degrees the inputs to the function and the axis and angle of rotation the outputs. Give the output axis as a unit vector. You may make the standard matrix for the composition of the rotations by multiplication of the standard matrices for the rotations about the position $x$-, $y$-, and $z$-axes, respectively. Using the function \textit{comp\_Rot}, check the result of the following commands:

\vspace{2mm}

\begin{verbatim}
>> [L ang]=comp_Rot(45, 45, 45)
\end{verbatim}

\end{exer}


\begin{sol}

\begin{verbatim}

%--- The following is the function file 'comp_Rot.m'. ---%
function [L, ang] = comp_Rot(ang1, ang2, ang3)

  % Angle as a degree.
  angle=[ang1 ang2 ang3];

  % Convert angles from degrees to radians.
  theta=angle*(pi/180); 

  % Rotation about x-axis, y-axis, and z-axis.
  Rx = [1 0 0; 
        0 cos(theta(1)) -sin(theta(1)); 
        0 sin(theta(1)) cos(theta(1))];
  Ry = [cos(theta(2)) 0 sin(theta(2)); 
        0 1 0; 
        -sin(theta(2)) 0 cos(theta(2))];
  Rz = [cos(theta(3)) -sin(theta(3)) 0; 
        sin(theta(3)) cos(theta(3)) 0; 
        0 0 1];

  % Composition of the three rotation matrices R = Ry*Rx*Rz.
  R = Rz*Ry*Rx; 

  % Find the axis of rotation of R,
  % by finding the eigenvector of R 
  % corresponding to the eigenvalue lambda=1.

  % Find a nonzero vector satisfying Rx = x.
  L = null(eye(3) - R);

  % Make the axis of rotation unit vector. 
  L = L/norm(L);  

  % Find the angle of rotation of R.
  % Note that w=(-x2,x1,0) is 
  % orthogonal to the axis of rotation x=(x1,x2,x3).
  w = [-L(2) L(1) 0]';
  rot_theta = acos((dot(w, R*w))/((norm(w)*norm(R*w))));
  ang = ((rot_theta)*(180/pi));
end
\end{verbatim}


\begin{outputs}

\begin{verbatim}

L =
   -0.3574
   -0.8629
   -0.3574

ang =
   64.7368
\end{verbatim}
\end{outputs}

\end{sol}
\newpage

\section{Geometry of Linear Operators}

\begin{exer} (\textit{Rotation as an Orthogonal Transformation})\\
Let
\begin{displaymath}
A = \left[\begin{array}{rrrr} \displaystyle -\frac{3}{7}&\quad \displaystyle -\frac{2}{7}&\quad \displaystyle -\frac{6}{7} \vspace{3mm} \\ \vspace{3mm} \displaystyle \frac{6}{7} & \displaystyle -\frac{3}{7} & \displaystyle -\frac{2}{7}\\ \displaystyle -\frac{2}{7} & \displaystyle -\frac{6}{7} & \displaystyle \frac{3}{7} \end{array} \right].
\end{displaymath}

\vspace{2mm}
\noindent Show that $A$ represents a rotation, and use Formulas $(16)$ and $(17)$ in Section $6.2$ to find the axis and angle of rotation.
\end{exer}

\begin{sol}
\begin{verbatim}

A = [-3/7 -2/7 -6/7; 6/7 -3/7 -2/7; -2/7 -6/7 3/7]; format short;

% Check that A*A' is the identity matrix.
A*A' 

% Although the off diagonal entries are not exactly all zeros,
% the scaling factor suggests that roundoff error prevents
% the computed matrix from being the identity matrix.
% You can see that the product is exactly the identity matrix,
% when the symbolic computation is used.

% Check that det(A)=1 to conclude that A is orthogonal.
det(A) 

% Since A*A'=I and det(A) = 1, A represents a rotation.

% By (16) of Section 6.2,
% find the angle of rotation 
% and convert the angle from radians to degrees.
theta = acos((trace(A)-1)/2); ang = ((theta)*(180/pi));
disp('The angle of rotation in degrees is'); disp(ang);

% By (17) of Section 6.2, find the axis of rotation.
% The initial point at the origin.
e1 = [1 0 0]'; 

% v is along the axis of rotation.
v = (A+A'+((1-trace(A))*eye(3))) * e1; 
disp('The axis of rotation passes through the point'); disp(v');
\end{verbatim}


\begin{outputs}

\begin{verbatim}

The angle of rotation in degrees is
  135.5847
The axis of rotation passes through the point
    0.5714    0.5714   -1.1429
\end{verbatim}
\end{outputs}
\end{sol}



\section{Kernel and Range}


\begin{exer}
(\textit{Invertible Matrix as a Bijective Linear Transformation})\\
Consider the matrix
\begin{displaymath}
A = \left[\begin{array}{rrrr} 3& \hspace{1mm} -5& \hspace{1mm} -2&\hspace{3mm} 2\\ -4 & 7 & 4 & 4 \\ 4 & -9 & -3 & 7 \\ 2 & -6 & -3 & 2 \end{array} \right].
\end{displaymath}

\vspace{2mm}
Referring to the Theorem $6.3.15$ in Section $6.3$, show that $T_{A} : \mathbb{R}^{4} \rightarrow \mathbb{R}^{4}$ is onto in at least four different ways. You may use several related MATLAB commands.

\end{exer}


\begin{sol}
\begin{verbatim}

A = [3 -5 -2 2; -4 7 4 4; 4 -9 -3 7; 2 -6 -3 2];
format short;

% By (a) in Theorem 6.3.15, use the command rref of A.
rref(A)

% Since the reduced row echelon form of A is the identity matrix,
% the linear transformation T is onto.

% By (d) in Theorem 6.3.15, use the command null of A.

% Find a basis for the null space of A.
null(A, 'r') 

% Since the null space contains only the zero vector,
% the linear transformation T is onto.

% By (g) in Theorem 6.3.15, use the command rank of A.

% Find the number of linearly independent columns of A.
rank(A) 

% Since the column vectors of A are linearly independent,
% the linear transformation T is onto.

% By (i) in Theorem 6.3.15, use the command det of A.

% Find the determinant of A.
det(A) 

% Since det(A) is nonzero,
% the linear transformation T is onto.

% By (j) in Theorem 6.3.15, use the command eig of A.

% Find the eigenvalues A.
eig(A) 

% Since 0 is not an eigenvalue of A,
% the linear transformation T is onto.
\end{verbatim}
\end{sol}



\section{Composition and Invertibility of Linear Transformations}



\begin{exer}(\textit{Compositions of Linear Transformations})\\
Consider successive rotations of $\mathbb{R}^{3}$ by $30\,^{\circ}$ about the $z$-axis, then by $60\,^{\circ}$ about the $x$-axis, and then by $45\,^{\circ}$ about the $y$-axis. If it is desired to execute the three rotations by a single rotation about an appropriate axis, what axis and angle should be used?

\end{exer}


\begin{sol}
\begin{verbatim}

% Angle as a degree.
ang1=30; ang2=60; ang3=45; 

% Convert angles from degrees to radians.
theta1=((ang1)*(pi/180)); 
theta2=((ang2)*(pi/180));
theta3=((ang3)*(pi/180)); 

format short;

% Rotation about z-axis with the angle 30.
Rz = [cos(theta1) -sin(theta1) 0; sin(theta1) cos(theta1) 0; 0 0 1];
% Rotation about x-axis with the angle 60.
Rx = [1 0 0; 0 cos(theta2) -sin(theta2); 0 sin(theta2) cos(theta2)];
% Rotation about y-axis with the angle 45.
Ry = [cos(theta3) 0 sin(theta3); 0 1 0; -sin(theta3) 0 cos(theta3)];

% Composition of the three rotation matrices R = Ry*Rx*Rz.
R = Ry*Rx*Rz; 

% Find the axis of rotation of R,
% by finding the eigenvector of R corresponding to the eigenvalue lambda=1.

% Find a nonzero vector satisfying Rx = x.
x = null(eye(3) - R); 

% Find the angle of rotation of R.
% Note that w=(-x2,x1,0) is orthogonal to the axis of rotation x=(x1,x2,x3).
w = [-x(2) x(1) 0]';
theta = acos((dot(w, R*w))/((norm(w)*norm(R*w))));

% Convert the angle from radians to degrees.
ang = ((theta)*(180/pi));

disp('The angle of rotation in degrees is'); disp(ang);
disp('The axis of rotation is'); disp(x');
\end{verbatim}


\begin{outputs}

\begin{verbatim}

The angle of rotation in degrees is
   69.3559

The axis of rotation is
   -0.9350   -0.3525   -0.0391
\end{verbatim}

\end{outputs}
\end{sol}